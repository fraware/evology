\documentclass{article}
\usepackage[utf8]{inputenc}
\usepackage{url}
\usepackage{hyperref}
\newcommand{\code}{\texttt}
\usepackage[
backend=biber,
style=alphabetic,
sorting=ynt
]{biblatex}
\addbibresource{bibliography.bib}

\title{Numerical experiments - market ecology}
\author{Aymeric Vi\'{e}}
\date{\today}

\begin{document}

\maketitle

\tableofcontents

\clearpage

\section{Current Model description}

\subsection{Initialisation}

We first create a population of $n$ trend-following agents, at time $t=0$. Each agent $i$ is described by its strategy $\theta_i(0)$, initial cash $C_i(0)$, initial assets $S_i(0)$. The $\theta$ parameter describes the time horizon on which the trend following strategy is based (Equation \ref{trend_following_equation}). Initial strategies are drawn uniformly in $(1,H]$, where $H$ is the maximum possible time horizon. \par
Values for initial endowments of the agents meet the $\$100M$ Securities and Exchange Commission threshold for institutional investment managers, that are equally split in cash and asset shares. 

\subsection{Market structure}

In this market, agents are choosing between a bond with a fixed continuously paid interest rate $r = 1\%$ annually, and an asset which pays dividends $D(t)$ at each step $t$, modelled as an autocorrelated geometric Brownian Motion.

\begin{equation}
    \label{dividend_equation}
    \begin{array}{l}
{D}(t)={D}(t)+g {D}(t-1)+\sigma {D}(t-1) {U}(t), \\
{U}(t)=\omega U(t-2)+\left(1-\omega^{2}\right){Z}(t)
\end{array}
\end{equation}

Where $g$ is the dividend growth rate, $\sigma$ its variance, $\omega$ is the autocorrelation parameter of the process, and $Z$ is a standard Wiener process. Based on estimates from Lebaron \cite{lebaron2001empirical} and Scholl et al. \cite{scholl2020market}, we take $g=2\%$ annually and $\sigma = 6\%$ annually. \par
The market is initialised with an initial price of $100$ that matches the fundamental value of the asset, and initial dividends of $100$. 

\subsection{Strategy and market behavior}
% here we write what the agents do, trading signal (write an example with a time series and a few different agents? that would be very nice. Inspire from David's figure 1?), excess demand, market clearing
% mention the next strategy extensions
% mention hypermutation & replacement

\begin{equation}
    \label{trend_following_equation}
    
\end{equation}

\begin{table}[]
    \centering
    \begin{tabular}{c|c|c}
    Mathematical notation & Meaning & Value \\
    \hline
        $n$ & Number of agents & 100 \\
        $t$ & Time index & $[1,t_{\text{max}}]$\\
        $\theta_i(t)$ & Agent strategy at time $t$ & $(1, H]$\\
        $C_i(0)$ & Initial cash & 50,000,000 \\
        $S_i(0)$ & Initial asset shares & 500,000 \\
        $D(t)$ & Dividend per asset share & See equation \ref{dividend_equation} \\
        $g$ & Dividend growth rate & $0.02$ annually \\
        $\sigma$ & Dividend variance & $0.06$ annually \\
        $\omega$ & Dividend autocorrelation & $0.1$ \\
    \end{tabular}
    \caption{Parameters and notations}
    \label{parameters_correspondence}
\end{table}

\subsection{Evolution of strategies}
% here we describe fitness as EMA, the GA operators, what the GA imitates (learning, imitation).

\section{Numerical experiment: identifying the impact of evolution in market dynamics}
% comparison between evolution and non evolution
% need a clear plan, not necessarily all finished, but at least for what's unclear (how many repetitions? How many agents?), some ideas to clear this out.



\section{Preamble: baseline configuration}

Necessary condition: running GA with complete market structure
Objective: be as close as possible to Maarten's paper configuration, notably for market environment parameters.
Add Agent leverage, aggressiveness of response to be closer to Maarten
\section{Preamble 2: observables}

Necessary condition: running GA with complete market structure
To implement:
\begin{itemize}
    \item Observe market dynamics over time (easy): price, returns, assets, volatility, dividends, profits
    \item Observe ecology dynamics over time (hard): proportions? heatmap? Add entropy?
    Perhaps H (for max time horizon) plots with clear color codes (like jet) over time, and hopefully we see something clear happen.
    \item Develop a GUI or something nice/updating itself over time to observe that?
\end{itemize}

\section{Trend follower market ecology}

Necessary condition: a running GA with a single asset, market clearing, and adequate starting conditions, and the baseline configuration.

\subsection{Influence of learning vs non-learning}
The non-lerning case (replacement dynamics only) can be illustrated with:
\begin{itemize}
    \item No selection (selection rate equal to 0)
    \item No crossover (crossover rate equal to 0)
    \item No mutation (mutation rate equal to 0)
\end{itemize}
The objective is to assess the behavior of the model in this no-learning setup, establish a benchmark.

Then, we can introduce learning at different degrees, and channels. Learning requires activation of selection.

\begin{enumerate}
    \item Mutation alone, with different values
    \item Crossover alone, with different values
    \item Mutation and crossover
\end{enumerate}

How does learning affect market dynamics? Ecology dynamics?

There are many many things we can also study just with trend followers:
\begin{itemize}
    \item Influence of strategy space with H
    \item Time horizon of EMA fitness profit
    \item Market parameters: dividend growth rate, volatility, interest rate, autocorrelation
    \item Alternative values for agent leverage, aggressiveness of response 
    \item Population size
    \item Tournament size
    \item Reinvestment rate
\end{itemize}

The key area of progress will be to add multiple strategy types, this may require a lot of work on the structure. 3 separate GAs with fixed population sizes? dynamic population sizes?

\printbibliography

\end{document}